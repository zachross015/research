\chapter{\sectioncite{rdp}}
\label{cha:rdp}

\section{$(\epsilon, \delta)$-Differential Privacy}%
\label{sec:_epsilon_delta_differential_privacy}


\begin{definition}
    \cite{dwork} A randomized algorithm $\cal{M}$ is $(\epsilon,
    \delta)$-\emph{differentially private} if for all $\cal{S} \subseteq
    \text{range}(\cal{M})$ and any two neighboring databases with a single modified
    record $D \sim D'$ has that
    \begin{equation}
        \pr{\cal{M}(D) \in \cal{S}} \leq e^\epsilon \pr{\cal{M}(D') \in \cal{S}}
        + \delta.
    \end{equation}
\end{definition}
\begin{definition}
    \cite{dwork} The $L_n$ sensitivity of a function $f$ $\left(\text{denoted }
    \Delta_n(f)\right)$, over any two
    neighboring databases $D, D'$ $\left(\text{denoted } D \sim D'\right)$ is
    \begin{equation}
        \Delta_n(f) = \sup_{D \sim D'}\norm{f(D) - f(D')}_n.
    \end{equation}
\end{definition}

\subsection{Mechanisms}%
\label{sub:mechanisms}

\begin{definition}
    The \emph{Randomized Response Mechansim} is defined for a predicate $f:
    \cal{D} \rightarrow \set{0, 1}$ as 
    \begin{equation}
        RR_p(f(D)) \defeq 
        \begin{cases}
            f(D) & \text{with probability } p, \\
            1 - f(D) & \text{with probability } 1 - p.
        \end{cases}
    \end{equation}
\end{definition}

\begin{definition}
    \cite{dwork} The \emph{Laplace Mechanism} with parameter $\epsilon$ is given
    by 
    \begin{equation}
        \cal{L}_\epsilon(f(\vec{X})) \defeq f(\vec{X}) + \left(Y_1, \dots, Y_n\right), 
    \end{equation}  where each $Y_i \sim \Lambda(0, \Delta_1(f) / \epsilon)$
    samples from the \emph{Laplace Distribution} which is
    characterized by the PDF
    \begin{equation}
        \Lambda(x | \mu, \sigma) = \frac{1}{2\sigma}\exp\left(- \frac{\abs{x -
        \mu}}{\sigma}\right).
    \end{equation}
\end{definition}

\begin{definition}
    \cite{dwork} The \emph{Gaussian Mechanism} with parameter $\sigma$ is given by
    \begin{equation}
        \cal{G}_\sigma\left(f\left(\vec{X}\right)\right) \defeq f\left(\vec{X}\right) +
        \cal{N}(0, \sigma^2 \vec{I}).
    \end{equation}
\end{definition}


\section{R\'enyi Differential Privacy}%
\label{sec:renyi_differential_privacy}


We can improve on the analysis given by $(\epsilon, \delta)$-DP using a
relaxation provided by R\'enyi Differential Privacy (RDP). This analysis
provides much stronger bounds for our analysis that the strong composition of
traditional DP.


\begin{definition}[R\'enyi divergence]
    For two probability distributions $P$ and $Q$ defined over $\cal{R}$, the
    \emph{R\'enyi divergence} of order $\alpha > 1$ is
    \begin{equation}
        D_\alpha (P\ ||\ Q) \defeq \frac{1}{\alpha - 1} \ln \bb{E}_{x \in
        Q}\left[{\left(\frac{P(x)}{Q(x)}\right)}^\alpha\right].
    \end{equation}
\end{definition}
\begin{definition}[$(\alpha, \epsilon)$-RDP]
    A randomized mechanism $\cal{M}: \cal{D} \rightarrow \cal{R}$ is said to
    have \emph{$\epsilon$-R\'enyi differential privacy} of order $\alpha$, or
    $(\alpha, \epsilon)$-RDP for short, if for any adjacent $D, D' \in \cal{D}$
    is holds that
    \begin{equation}
        D_\alpha (\cal{M}(D)\ ||\ \cal{M}(D')) \leq \epsilon.
    \end{equation}
\end{definition}
\begin{proposition}[Composition of RDP]
    Let ${f}: \cal{D} \rightarrow \cal{R}_1$ be $(\alpha, \epsilon_1)$-RDP
    and ${g}: \cal{R}_1 \times \cal{D} \rightarrow \cal{R}_2$ be $(\alpha,
    \epsilon_2)$-RDP. Then the mechanism defined as $(\cal{M}_1, \cal{M}_2)$,
    where $\cal{M}_1 \sim f(\cal{D})$ and $\cal{M}_2 \sim g(\cal{M}_1,
    \cal{D})$, satisfies $(\alpha, \epsilon_1 + \epsilon_2)$-RDP.
\end{proposition}

\begin{definition}[$c$-stable Transformations]
    We say that $g: \cal{D} \rightarrow \cal{D}'$ is $c$-stable if $g(A)$ and
    $g(B)$ are adjacent in $\cal{D}'$ implies that there exists a sequence of
    length $c + 1$ so that $D_0 = A, \dots, D_c = B$ and all $(D_i, D_{i + 1})$
    are adjacent in $\cal{D}$.
\end{definition}

\begin{proposition}[$2^c$-stable RDP]
    If $f: \cal{D} \rightarrow \cal{R}$ is $(\alpha, \epsilon)$-RDP, $g:
    \cal{D}' \rightarrow \cal{D}$ is $2^c$-stable and $\alpha \geq 2^{c+1}$,
    then $f \circ g$ is $(\alpha / 2^c, 3^c \epsilon)$-RDP.
\end{proposition}

\subsection*{Randomized Response RDP}%
\label{sub:mechanism_rdp}

\begin{proposition}
    The randomized response mechanism with probability $p$ satisfies
    \begin{equation}
        \left(\alpha, \frac{1}{\alpha - 1}\log\left(p^\alpha{(1 - p)}^{1 -
        \alpha} + {(1 - p)}^\alpha p^{1 - \alpha}\right)\right)-\text{RDP}
    \end{equation} if $\alpha > 1$, and 
    \begin{equation}
        \left(\alpha, (2p - 1)\log\left(\frac{p}{1 - p}\right)\right)-\text{RDP}
    \end{equation} if $a = 1$.
\end{proposition}

\subsection*{Laplacian RDP}%
\label{sub:laplacian_rdp}

\begin{proposition}
    For any $\alpha \geq 1$ and $\lambda > 0$:
    \begin{equation}
        D_\alpha(\Lambda(0, \lambda)\ ||\ \Lambda(1, \lambda)) = \frac{1}{\alpha
        - 1} \log \left\{\frac{\alpha}{2\alpha - 1}\exp\left(\frac{\alpha -
        1}{\lambda}\right) + \frac{\alpha - 1}{2\alpha - 1}\exp
        \left(\frac{-\alpha}{\lambda}\right)\right\}.
    \end{equation}
\end{proposition}

\begin{corollary}
    If real-valued function $f$ has sensitivity 1, then the Laplace Mechanism
    $\cal{L}_\lambda(f)$ satisifies $\left(\alpha, \frac{1}{\alpha
        - 1} \log \left\{\frac{\alpha}{2\alpha - 1}\exp\left(\frac{\alpha -
        1}{\lambda}\right) + \frac{\alpha - 1}{2\alpha - 1}\exp
        \left(\frac{-\alpha}{\lambda}\right)\right\}\right)$-RDP.
\end{corollary}


\subsection*{Gaussian RDP}%
\label{sub:gaussian_rdp}


\begin{proposition}
        $D_\alpha(\cal{N}(0, \sigma^2)\ ||\ \cal{N}(\mu, \sigma^2)) =
        {\alpha\mu^2}/{2 \sigma^2}$
\end{proposition}
\begin{corollary} 
    \label{cor:dpgauss}
    The gaussian mechanism $\cal{G}_\sigma$ applied to a function $f$ satisfies $(\alpha, \alpha
    {\Delta_2(f)}^2 / (2\sigma^2))$-RDP.
\end{corollary}

\noindent Finally, we can recover the $\epsilon$ and $\delta$ terms from RDP.

\begin{proposition}
    If $\cal{M}$ is an $(\alpha, \epsilon)$-RDP mechanism, it also satisfies
    $\left(\epsilon + \frac{\ln 1 / \delta}{\alpha - 1}, \delta\right)$-DP for any $0
    < \delta < 1$.
\end{proposition}


