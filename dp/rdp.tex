\chapter{\sectioncite{rdp}}
\label{cha:rdp}

\section{$(\epsilon, \delta)$-Differential Privacy}%
\label{sec:_epsilon_delta_differential_privacy}


\begin{definition}
    \cite{dwork} A randomized algorithm $\cal{M}$ is $(\epsilon,
    \delta)$-\emph{differentially private} if for all $\cal{S} \subseteq
    \text{range}(\cal{M})$ and any two neighboring databases with a single modified
    record $D \sim D'$ has that
    \begin{equation}
        \pr{\cal{M}(D) \in \cal{S}} \leq e^\epsilon \pr{\cal{M}(D') \in \cal{S}}
        + \delta.
    \end{equation}
\end{definition}
\begin{definition}
    \cite{dwork} The \emph{Gaussian Mechanism} with parameter $\sigma$ is given by
    \begin{equation}
        \cal{G}_\sigma\left(f\left(\vec{X}\right)\right) = f\left(\vec{X}\right) +
        \cal{N}(0, \sigma^2 \vec{I}).
    \end{equation}
\end{definition}
\begin{definition}
    \cite{dwork} The $L_2$ sensitivity of a function $f$ $\left(\text{denoted }
    \Delta_f\right)$, over any two
    neighboring databases $D, D'$ $\left(\text{denoted } D \sim D'\right)$ is
    \begin{equation}
        \Delta_f = \sup_{D \sim D'}\norm{f(D) - f(D')}_2.
    \end{equation}
\end{definition}
% \begin{theorem}
%     \label{thm:dpgauss}
%     \textnormal{\cite{dwork}} Let $\epsilon \in (0, 1)$ be arbitrary. For $a^2 > 2 \ln(1.25 / \delta)$,
%     the Gaussian Mechanism with parameter $\sigma \geq a \Delta_f/\epsilon$
%     is $(\epsilon, \delta)$-differentially private.
% \end{theorem}


\section{R\'enyi Differential Privacy}%
\label{sec:renyi_differential_privacy}


We can improve on the analysis given by $(\epsilon, \delta)$-DP using a
relaxation provided by R\'enyi Differential Privacy (RDP). This analysis
provides much stronger bounds for our analysis that the strong composition of
traditional DP.


\begin{definition}[R\'enyi divergence]
    For two probability distributions $P$ and $Q$ defined over $\cal{R}$, the
    \emph{R\'enyi divergence} of order $\alpha > 1$ is
    \begin{equation}
        D_\alpha (P\ ||\ Q) \defeq \frac{1}{\alpha - 1} \ln \bb{E}_{x \in
        Q}\left[{\left(\frac{P(x)}{Q(x)}\right)}^\alpha\right].
    \end{equation}
\end{definition}
\begin{definition}[$(\alpha, \epsilon)$-RDP]
    A randomized mechanism $\cal{M}: \cal{D} \rightarrow \cal{R}$ is said to
    have \emph{$\epsilon$-R\'enyi differential privacy} of order $\alpha$, or
    $(\alpha, \epsilon)$-RDP for short, if for any adjacent $D, D' \in \cal{D}$
    is holds that
    \begin{equation}
        D_\alpha (\cal{M}(D)\ ||\ \cal{M}(D')) \leq \epsilon.
    \end{equation}
\end{definition}
\begin{proposition}[Composition of RDP]
    Let ${f}: \cal{D} \rightarrow \cal{R}_1$ be $(\alpha, \epsilon_1)$-RDP
    and ${g}: \cal{R}_1 \times \cal{D} \rightarrow \cal{R}_2$ be $(\alpha,
    \epsilon_2)$-RDP. Then the mechanism defined as $(\cal{M}_1, \cal{M}_2)$,
    where $\cal{M}_1 \sim f(\cal{D})$ and $\cal{M}_2 \sim g(\cal{M}_1,
    \cal{D})$, satisfies $(\alpha, \epsilon_1 + \epsilon_2)$-RDP.
\end{proposition}

\begin{definition}[$c$-stable Transformations]
    We say that $g: \cal{D} \rightarrow \cal{D}'$ is $c$-stable if $g(A)$ and
    $g(B)$ are adjacent in $\cal{D}'$ implies that there exists a sequence of
    length $c + 1$ so that $D_0 = A, \dots, D_c = B$ and all $(D_i, D_{i + 1})$
    are adjacent in $\cal{D}$.
\end{definition}

\begin{proposition}[$2^c$-stable RDP]
    If $f: \cal{D} \rightarrow \cal{R}$ is $(\alpha, \epsilon)$-RDP, $g:
    \cal{D}' \rightarrow \cal{D}$ is $2^c$-stable and $\alpha \geq 2^{c+1}$,
    then $f \circ g$ is $(\alpha / 2^c, 3^c \epsilon)$-RDP.
\end{proposition}

\noindent We want to establish the use of the gaussian mechanism in conjunction with RDP,
since the gaussian mechanism has the additive property, making it simpler for
further analysis. To do so, we first make note of the following proposition.

\begin{proposition}
        $D_\alpha(\cal{N}(0, \sigma^2)\ ||\ \cal{N}(\mu, \sigma^2)) =
        {\alpha\mu^2}/{2 \sigma^2}$
\end{proposition}
\begin{corollary} 
    \label{cor:dpgauss}
    The gaussian mechanism $\cal{G}_\sigma$ applied to a function $f$ satisfies $(\alpha, \alpha
    \Delta_f^2 / (2\sigma^2))$-RDP.
\end{corollary}

\noindent Finally, we can recover the $\epsilon$ and $\delta$ terms from RDP.

\begin{proposition}
    If $\cal{M}$ is an $(\alpha, \epsilon)$-RDP mechanism, it also satisfies
    $\left(\epsilon + \frac{\ln 1 / \delta}{\alpha - 1}, \delta\right)$-DP for any $0
    < \delta < 1$.
\end{proposition}


