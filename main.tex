\documentclass[oneside]{book}

\usepackage{zross-style}
\usepackage{booktabs}
\usepackage{hyperref}
\usepackage[b5paper, margin=1in]{geometry}
\usepackage{import}


\addbibresource{bib.bib}

% References 
\newcommand{\sectioncite}[1]{\citefield{#1}{title}~\cite{#1}}
\renewcommand{\chaptermark}[1]{\markboth{\chaptername\ \thechapter.\ #1}{}}


% Commonly used variables
\newcommand{\crm}{\rho_{\max}}
\newcommand{\crcc}{\rho_{\text{cr}}}
\newcommand{\thetaprune}{\theta_{\text{prune}}}
\renewcommand{\th}{\en{\text{th}}}


% Pruning paper
\newcommand{\synapsemetric}[1]{\en{\dotp{\pdv{\cal{R}}{#1}, #1}}}


% Research write up
\renewcommand{\L}{\cal{L}}
\newcommand{\z}{\vec{z}}
\newcommand{\h}{\vec{h}}
\newcommand{\W}{\mat{W}}


\title{General Research Notes}
\author{Zachary Ross}

\begin{document}


\maketitle

\tableofcontents
\pagebreak


\part{Dated Findings}

\subsectiondate{2022-02-02}%

In the work presented to me, Dr. Lee primarily uses pruning for gradient masking
rather than parameter masking, although most studies I've read seem to have done
the opposite. I initially assumed that given the standard linear layer function
with masking 
\begin{equation}
    \label{eq:linearmask}
    \vec{z} = (\mat{M} \odot\mat{W})\vec{h} + \vec{b}
\end{equation} the effect of the mask would still result in non-zero values in
the gradient of the matrix. This is falsified by the derivation
\begin{equation}
        \pdv{L}{W_{i,j}} = \pdv{L}{z_i}\pdv{z_i}{W_{ij}} = M_{ij} \pdv{L}{z_i}
        h_j
\end{equation} or in vectorized form we have that 
\begin{equation}
    \label{eq:backgrad}
    \pdv{L}{\mat{W}} = \mat{M} \odot \left(\pdv{L}{\vec{z}}\tpose{\vec{h}}\right)
\end{equation} which implies that the use of a forward mask guarantees that of a
backward. The method used instead in Dr. Lee's examples was $\vec{z} =
\mat{W}\vec{h} + \vec{b}$ with masking only applied during the gradient (i.e.\
that of \refeq{backgrad}). So what needs to be tested is whether or not we can
apply masking solely during the forward phase i.e.\ implement \refeq{linearmask}
with builds its gradient without the mask.


\subsectiondate{2022-02-09}


Thorough testing has disproved our previous hypothesis. Strangely enough, the
gradient masking is much more effective than weight masking for some odd reason.


\subsectiondate{2022-02-23}

In the following discussion, I use beats per minute (BPM) to refer to heart
rate. I only do it this way because I've already written it out and don't feel
like changing it.

I've been pondering a bit recently on a hypothesis I've had, that being the
perception of time is relative to the individual based on several factors. At
first I believed that it's connected to BPM, but this was disproved by my
addiction to coffee. I've noticed this relativity even in my day to day, some
days seem faster than others while certain activities seem to be slower. For
example, under the influence of coffee I feel that my day goes by much faster,
which leads me to believe the effects of caffeine are misunderstood (which I
will go into later). After or during an intense workout, my perception of time
seems to slow way the hell down. I notice this because I'll be listening to
music, and maybe it has to do with my percussionist background with intense
focus on steady timing, but the songs pass by much slower. This is incredibly
curious. 

On other days where I don't quite feel myself, I notice instead that the
music tends to be much quicker, often times leading me to feel like I'm missing
alot. 

For the coffee thing, maybe its affect on BPM does slow down perception of
time, but caffeine could effectively and periodically cut out clips of time,
making us believe that time is moving faster. I believe this is affects
productivity because it allows us to move from task to task seemingly "quicker".
I don't believe it fully alters our productivity; rather, it seems like it
allows us to reach the reward much quicker and thereby reinforce the thought
process that we are "getting shit done". This activity/reward behavior most
likely becomes a self-fulfilling cycle as the momentum continues, motivating us
to continue doing tasks.

I've also considered how we would test these hypotheses because it's rather
elusive. Say we tried to get a set of people to listen to a meteronome prior to
an energy intensive activity and post said activity and asked them to compare
the two for which one is faster. For one, how would we get reliable information?
If we were just to pick random people from a population, they may not even be
capable of discerning between two tempos. I suppose we could do a preliminary
exam where we vet out those who could not tell the difference even between minor
changes in tempo. We could pick out trained musicians, but then the study is
limited to musicians techinically. I guess we would just have to do the vetting
procedure and only select the upper echelon of people which can tell extremely
minute differences in tempo after a select period of time. 

Thoughts I had while working out, lol.

\import{}{directories}


\printbibliography


\end{document}
