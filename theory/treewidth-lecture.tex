\section{\sectioncite{treewidth-lecture}}
\label{cha:treewidth-lecture}

\begin{definition}
    \label{def:treedecomp}
    A \emph{tree decomposition} of a graph $G = (V, E)$ is a pair 
    \begin{equation}
        (\set{X_i\ |\ i \in I}, T = (I, M))
    \end{equation}
    where $\set{X_i\ |\ i \in I}$ is a collection of subsets $V$ (also called
    \emph{bags}), and $T$ is a tree s.t. 
    \begin{itemize}
        \item $\cup_{i \in I} X_i = V$ (combining the vertices of all bags forms
            $V$)
        \item $(u, v) \in E \implies \exists i \in I : u, v \in X_i$ (if
            there is an edge connecting two vertices, then those vertices share
            at least one bag)
        \item $\forall v\in V: \set{i \in I\ |\ v \in X_i}$ induces a connected
            subtree of $T$ (all bags involving a vertex $v$ form a connected
            subtree).
    \end{itemize}

    Similarly, a \emph{path decomposition} is a tree decomposition such that $T$
    is a path (no branches).
\end{definition}

\begin{remark}
    The last statement in \refdef{treedecomp} can be replaced with 
    \begin{itemize}
        \item $i, k, j \in I$ and $j$ is on the path from $i$ to $k$ in $T
            \implies X_i \cap X_k \subseteq X_j$.
    \end{itemize}
\end{remark}

\begin{lemma} \textnormal{\cite{bodlander}}
    Let $(\set{X_i\ |\ i \in I}, T = (I, M))$ be a tree decomposition of $G =
    (V, E)$, and let $K \subseteq V$ be a clique in $G$. Then $\exists i \in I
    : K \subseteq X_i$.
\end{lemma}

\begin{definition}
    The \emph{width} of a decomposition is $\max_{i \in I} \abs{X_i} - 1$. The
    \emph{tree-width} ($tw(\cdot)$) is the minimum width over all tree
    decompositions of $G$. Likewise, the \emph{path-width} ($pw(\cdot)$) is the
    minimum width over all path decompositions of $G$.
\end{definition}

\begin{corollary}
    For every graph, $tw(G) \geq \omega(G) - 1$ ($\omega(G)$ is the size of the
    max clique in $G$).
\end{corollary}


\begin{theorem}\textnormal{\cite{ktreeembed}}
    The following problems are NP-complete:
    \begin{itemize}
        \item Given a graph $G = (V, E)$ and an integer $k < \abs{V}$, is the
            $tw(G) \leq k$?
        \item Given a graph $G = (V, E)$ and an integer $k < \abs{V}$, is the
            $pw(G) \leq k$?
    \end{itemize}
\end{theorem}

\begin{definition}
    The class of $K$-trees is defined recursively as follows:
    \begin{itemize}
        \item $K_{k + 1}$ is a $k$-tree;
        \item a $k$-tree $G$ with $n + 1$ vertices ($n \geq k+1$) can be
            constructed from a $k$-tree $H$ with $n$ vertices by adding a vertex
            adjacent to exactly $k$ vertices, namely all vertices of a $(k +
            1)$-clique of $H$.
    \end{itemize}
\end{definition}

\begin{theorem} \textnormal{\cite{ktreechar}}
    Let $G = (V, E)$ be a graph. The following statements are equivalent:
    \begin{itemize}
        \item $G$ is a $k$-tree.
        \item $G$ is a conneted, $G$ has a $k$-clique, but no $(k + 2)$-cliques,
            and every minimal separator of $G$ is a $k$-clique.
        \item $G$ is connected, $\abs{E} = k \abs{V} - \frac{1}{2}k(k + 1)$, and
            every minimal separator of $G$ is a $k$-clique.
        \item $G$ has a $k$-clique, but not a $(k + 2)$-clique, and every
            minimal separator of $G$ is a clique, and for all distinct
            non-adjacent pairs of vertices $x, y \in V$, there exist exactly $k$
            vertex disjoint paths from $x$ to $y$.
    \end{itemize}
\end{theorem}

\begin{definition}
    A \emph{partial $k$-tree} is a grpaph that contains all the vertices and a
    subset of the edges of a $k$-tree.
\end{definition}

\begin{theorem}
    \textnormal{\cite{van1990graph}}
    $G$ is a partial $k$-tree if and only if $G$ has tree-width at most $k$.
\end{theorem}

