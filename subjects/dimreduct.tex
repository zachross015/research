%%%%%%%%%%%%%%%%%%%%%%%%%%%%%%%%%%%%%
% CHAPTER: DIMENSIONALITY REDUCTION %
%%%%%%%%%%%%%%%%%%%%%%%%%%%%%%%%%%%%%

\chapter{Dimensionality Reduction}%
\label{cha:dimensionality_reduction}


\section{Random Projections}%
\label{sec:random_projections}


\begin{definition}
    \cite{scalable-opt} \emph{Random projections} are low-dimensional embeddings $\mat{\Pi}: \R^\rho
    \rightarrow \R^\tau$ which preserve - up to a small distortion - the
    geometry of a subspace of vectors.
\end{definition}

For a matrix $\mat A \in \R^{m \times n}$ with $\rank{\mat A} = r$, we consider
the block SVD for some $1 \leq k \leq r$ given by 
\begin{equation}
    \mat A = \bmat{\mat U_k & \mat U_{r - k}}
    \bmat{\mat \Sigma_k & \\ & \mat \Sigma_{r - k}} 
    \bmat{\mat V^\intercal_k \\ \mat V^\intercal_{r - k}}
\end{equation} where the singular values are ordered s.t. $\sigma_1 \geq \cdots
\geq \sigma_k \geq \sigma_{k + 1} \geq \cdots \geq \sigma_r > 0$. We also denote
$\mat A_k = \mat U_k \mat \Sigma_k \mat V_k^\intercal$. 

We now define a list of random projections that have been observed in papers. We
first consider a class of random projections having the following form:
\begin{equation}
    \label{eq:rp}
    \mat{\Pi} = \sqrt{\rho / \tau}\mat{S} \mat{\Theta} \mat{D}
\end{equation} where 
\begin{enumerate}
    \item $\rho$ is the original dimensionality,
    \item $\tau$ is the reduced dimensionality,
    \item $\mat{S} \in \R^{\tau \times \rho}$ is a subsampling matrix,
    \item $\mat \Theta \in \R^{\rho \times \rho}$ is a structured orthonormal transformation, and
    \item $\mat{D} \in \R^{\rho \times \rho}$ is a diagonal matrix whose entries
        are drawn independently from $\set{-1, 1}$,
\end{enumerate}

\begin{definition}
    \cite{improved-matrix-algorithms-via-the-subsampled-randomized-hadamard-transform}
    Let $\rho = 2^n$. The \emph{Non-normalized Walsh-Hadamard Transform} is a block matrix of
    dimensionality $\rho \times \rho$, defined recursively as 
    \begin{equation}
        \mat H_2 = \bmat{1 & 1 \\ 1 & -1}, \quad \text{and} \quad 
        \mat H_{\rho} = \bmat{\mat H_{\rho / 2} & \mat H_{\rho / 2} \\ \mat
        H_{\rho / 2} & - \mat H_{\rho / 2} }.
    \end{equation} The normalized variation is defined as $\rho^{-1/2}\mat
    H_{\rho}$.
\end{definition}

\begin{definition}
    The \emph{Subsampled Random Hadamard Transform} (SRHT) is a structured random
    projection given by \refeq{rp} where $\mat{\Theta} \in \R^{\rho \times \rho}$ is a
    normalized Walsh-Hadamard Transform.
\end{definition}

The primary issue with using SRHT is it requires the numbers of parameters to be
a power of two; however, when this is possible, then $\mat \Pi$ can be
constructed and $\mat \Pi \vec x$ can be computed in $\cal O(2\rho\lg(\tau +
1))$ operations
\cite{fast-dimension-reduction-using-rademacher-series-on-dual-bch-codes}. 

\begin{definition}
    The \emph{Subsampled Randomized Fourier Transform} (SRFT) is a \emph{structured}
    random projection given by \refeq{rp} where $\mat{\Theta} \in \R^{\rho \times
    \rho}$ is a unitary discrete fourier transform matrix.
\end{definition}

The SRFT alleviates the size constraints of the SRHT. The paper I have been
reviewing so far
\cite{improved-matrix-algorithms-via-the-subsampled-randomized-hadamard-transform}
has only performed an analysis on the matrix multiplication case, so I will need
to find another on the vector projection case.
