\subsection{An Introduction to Quantitative Finance}

\subsubsection{Interest Rates}

\begin{definition}
    A \emph{notional} or \emph{principal} is an initial deposit or value $N$.
\end{definition}

\begin{definition}
    \label{def:irate}
    An \emph{interest rate} $r$ is the rate at which a value is increased according to
    a specified frequency of time. 
\end{definition}

Suppose we have an account with interest rate $r$ and notional $N$. The value of 
the account at time $T$ is 
\begin{equation}
    N_T = Ne^{rT}
\end{equation}
Throughout finance, there exist discrete and continuous analogs for most
equations due the discrete case being more realistic while the  continuous case
is far more mathematically efficient. That being said, discrete analogs will be
used sparsely and we will primarily depend on continuous equations. The discrete
versions can be derived from the continuous versions by noting that
\begin{equation}
    e^{rT} = {\left(1 + \frac{r_m}{m}\right)}^{mT}
\end{equation} where $T$ is commonly denoted as the time in years, $m$ is the
frequency the interest is compounded, and $r_m$ is the equivalent interest rate
with the discrete frequency. $r_m$ can then be recovered from $r$ using
\begin{equation}
    r_m = m\left(e^{\left(\frac{r}{m}\right)} - 1\right).
\end{equation}

\begin{definition}
    A \emph{money market account} is an account compounded continuously at rate
    $r$ with notional 1. I.e. $M_0 = 1$ and $M_{t} = e^{rt}$.
\end{definition}

\begin{definition}
    A \emph{zero coupon bond} (ZCB) with maturity $T$ is an asset that pays 1 at
    time $T$ (and nothing else). The value of a ZCB at time $t$ for a
    continuously compounded rate $r$ is denoted as 
    \begin{equation}
        \label{eq:zcb}
        Z(t, T) = e^{-r(T-t)}
    \end{equation} where $Z(T, T) = 1$ by definition.     
    The values $Z(t, T)$ with $0 \leq t \leq T$ are known as \emph{discount
    factors} or \emph{present values}.
\end{definition}

The intuition behind the present values for a ZCB derives from the fact that if
we have two portfolios, one a ZCB with maturity $T$ and another that contains
cash which will accumulate interest to be worth 1 at time $T$, then both will be
worth the same price at each $t$ leading up to $T$ and thereafter.
