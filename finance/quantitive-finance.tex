\section{\sectioncite{blyth}}

It is important to note that I only present the conclusions to propositions
presented in this book rather than deriving the proofs themselves. This is
especially the case whenever I list out the value of any derivative: these
values are derived in the book using various proofs and are \emph{not} defined
to have these values in the first place. For the most part, any derivative with
maturity $T$ has a definitive value at maturity, but is a random variable up
until that point. The book spends a good amount of time providing deterministic
values for these r.v.s, and all that is generally listed here are these
deterministic values.

\subsection{Interest Rates}

\begin{definition}
    A \emph{notional} or \emph{principal} is an initial deposit or value $N$.
\end{definition}

\begin{definition}
    \label{def:irate}
    An \emph{interest rate} $r$ is the rate at which a value is increased
    according to a specified frequency of time. 
\end{definition}

Suppose we have an account with interest rate $r$ and notional $N$. The value of 
the account at time $T$ is 
\begin{equation}
    N_T = Ne^{rT}
\end{equation}
Throughout finance, there exist discrete and continuous analogs for most
equations due the discrete case being more realistic while the  continuous case
is far more mathematically efficient. That being said, discrete analogs will be
used sparsely and we will primarily depend on continuous equations. The discrete
versions can be derived from the continuous versions by noting that
\begin{equation}
    \label{eq:rtone}
    e^{rT} = {\left(1 + \frac{r_m}{m}\right)}^{mT}
\end{equation} where $T$ is commonly denoted as the time in years, $m$ is the
frequency the interest is compounded, and $r_m$ is the equivalent interest rate
with the discrete frequency. $r_m$ can then be recovered from $r$ using
\begin{equation}
    \label{eq:rttwo}
    r_m = m\left(e^{\left(\frac{r}{m}\right)} - 1\right).
\end{equation}

\begin{definition}
    A \emph{money market account} is an account compounded continuously at rate
    $r$ with notional 1. I.e. $M_0 = 1$ and $M_{t} = e^{rt}$.
\end{definition}

\begin{definition}
    A \emph{zero coupon bond} (ZCB) with maturity $T$ is an asset that pays 1 at
    time $T$ (and nothing else). The value of a ZCB at time $t$ for a
    continuously compounded rate $r$ is denoted as 
    \begin{equation}
        \label{eq:zcb}
        Z(t, T) = e^{-r(T-t)}
    \end{equation} where $Z(T, T) = 1$ by definition.     
    The values $Z(t, T)$ with $0 \leq t \leq T$ are known as \emph{discount
    factors} or \emph{present values}.
\end{definition}

The intuition behind the present values for a ZCB derives from the fact that if
we have two portfolios, one a ZCB with maturity $T$ and another that contains
cash which will accumulate interest to be worth 1 at time $T$, then both will be
worth the same value at each $t$ leading up to $T$ and thereafter.

\begin{definition}
    An \emph{annuity} a series of fixed cashflows at specified times $T_i = 1,
    \dots, n$. The value at time $t$ is given by 
    \begin{equation}
        \label{eq:annuity}
        V = C\sum_{i = 1}^n Z(t, T_i).
    \end{equation}
\end{definition}
When we substitute in \refeq{zcb}, we notice that this is just a geometric
series. Because of this, if an annuity is payed once a year for $M$ years with
rate $r$, this
equates to 
\begin{equation}
    V = \sum_{i = 1}^n \frac{1}{{(1 + r)}^i} = \frac{1}{r}\left(1 - \frac{1}{{(1 + r)}^M}\right)
\end{equation}

\begin{definition}
    The \emph{accrual factor} is (under the discrete analogy) the frequency
    which interest is compounded, and is denoted as $\alpha$.
\end{definition}

\begin{table}[t]
    \centering
    \caption{Daycount conventions and accrual factors}
    \label{tab:dccon}
    \begin{tabular}{ll}
        \toprule
        \textbf{Daycount convention} & \textbf{$\alpha$ for 16 December 2011 -
        16 March 2012} \\
        \midrule
        act/365 & 91/365 \\
        act/act & 15/365 + 76/366 \\
        act/360 & 91/360 \\
        30/365 & 1/4 \\
        \bottomrule
    \end{tabular}
\end{table}

In Equations~\ref{eq:rtone} and~\ref{eq:rttwo}, this is the same as taking
$\alpha = 1/m$. Due to the year being uneasily divided by simple values of
$\alpha$, several \emph{daycount conventions} have been made for different
settings. We list a sample of these in Table~\ref{tab:dccon}. The daycount
conventions are seemingly deceptive in that they hide various details behind
terminology. For example, two cases are 30/x and x/360, which respectively
denote that each month in the period should be counted as only having 30 days
and each month in a year should be counted as only having 30 days. I want to
emphasize the weird convention because when we encounter a fraction in the work
such as 30/act, this means to say that if we specify a period of three months,
then this fraction actually denotes 
\begin{equation}
    \frac{30 * 3}{m_1 + m_2 + m_3}
\end{equation} where each $m_i$ is the number of actual days in the specified
month. Like wise, if we have a daycount act/act, this denotes the number specifies
the number of actual days in the period divided by the actual days in the year.
If we have a leap year, then the denominator contains 366, whereas if February
is included in the daycount, then we may have 28 or 29 days dependent on the
year. I hope this helps to explain it better, if not then \emph{oof}.

A few specific cases arise but are only defined in the exercises. I will list
those here.

\begin{definition}
    \emph{Semi-bond} is the US dollar interest rate which is semi-annual
    compounding with 30/360 daycount and is denoted as $\gamma_{SB}$.
\end{definition}

\begin{definition}
    \emph{Annual money} is the US dollar interest rate which is annual
    compounding with act/360 daycount and is denoted as $\gamma_{AM}$.
\end{definition}

If we know the rate $r_{\alpha}$ and accrual factor $\alpha$ of a quoted
interest rate over a specified period of time, and we are interested in the
equivalent rate $r_\beta$ under a different accrual factor $\beta$ over the same
period of time, then this information can be retrieved by making note of the
equivalency 
\begin{equation}
    r_\alpha\alpha = r_\beta\beta.
\end{equation} This makes conversions simple as demonstrated by the following
example. 

\begin{example}
    If on 16 December 2011 the act/365 rate for three months is $5\%$, the
    act/360 rate is 
    \begin{equation}
        r_\beta = r_\alpha\frac{\alpha}{\beta} = 5\% \frac{\text{act/365}}
        {\text{act/360}} = 5\%\frac{360}{365} = 4.9315\%.
    \end{equation}
\end{example}

\begin{example}
    The conversion between semi-bonds and annual money is given by 
    \begin{equation}
        \gamma_{SB}\frac{30}{360} = \gamma_{AM}\frac{\text{act}}{360},
    \end{equation}
    so that 
    \begin{equation}
        \gamma_{SB} = \gamma_{AM}\frac{\text{act}}{30},
    \end{equation} and likewise
    \begin{equation}
        \gamma_{AM} = \gamma_{SB}\frac{30}{\text{act}}.
    \end{equation}
\end{example}

\begin{definition}
    A \emph{stock} or \emph{share} is an asset giving ownership in a fraction of
    a company. The price at time $t$ of a stock is denoted by $S_t$. The current
    known price is called the \emph{spot}.
\end{definition}

\begin{definition}
    A \emph{fixed rate bond} with coupon $c$ and notional $N$ is an asset that
    pays a coupon $cN$ at a specified frequency (usually a year) and $N$ at its
    maturity date $T$. A \emph{floating rate bond} is defined similarly with a
    variable interest rate.
\end{definition}


\subsection{Forward Contracts and Forward Prices}%
\label{sec:forward_contracts_and_forward_prices}


\begin{definition}
    A \emph{derivative contract} (\emph{derivative}) is a financial
    contract between two parties whose value derives from the value of another
    variable.
\end{definition}

\begin{remark}
    Derivatives are often defined by the payout at a specified time. E.g.\ if
    $S_T$ indicates the total snowfall in inches at time $T$ then the weather
    derivative $g(S_T)$ can be defined by
    \begin{equation}
        g(S_T) = I\set{S_T > 50} = \begin{cases}
            \$1 &\text{if } S_T > 50 \\
            0 &\text{otherwise},
        \end{cases}
    \end{equation} 
    where both $S_T$ and $g(S_T)$ are random variables with unknown values until
    $T = 1$.
\end{remark}

\begin{definition}
    A \emph{forward contract} (\emph{forward}) is an agreement between two
    counterparties to trade a specific asset at maturity $T$ with delivery price
    $K$. At a time $t \leq T$, the buyer is \emph{long} the contract while the
    seller is \emph{short} the contract. 
\end{definition}

The value at time $t \leq T$ of being long a forward contract is given by
$V_K(t, T)$, where by construction we have that $V_K(T, T) = S_T - K$.

\begin{definition}
    The \emph{forward price} $F(t, T)$ at current time $t \leq T$ is the
    delivery price $K$ s.t.\ $V_K(t, T) = 0$.
\end{definition}

\begin{remark}
    The forward price can be seen as denoting the spot price at maturity, so in
    essence you will be agreeing at time $t \leq T$ to trade the asset at time
    $T$ for its actual price. If you had $K < F(t, T)$ then the person long the
    contract has a better deal since they would pay less than the appreciated
    value (i.e.\ you could just sell the asset for profit), whereas $K > F(t,
    T)$ provides the better deal for the short since they would receive more
    than the appreciated value.
\end{remark}

\begin{table}[h]
    \centering
    \caption{Forward prices for various assets}
    \label{tab:forwardprices}
    \begin{tabular}{lr}
        \toprule
        \textbf{Underlying} & \textbf{Forward price} \\
        \midrule
        Asset paying no income & ${S_t}e^{r(T-t)}$ \\
        Asset paying known income $I$ & $({S_t} - I)e^{r(T-t)}$ \\
        Asset paying dividends at rate $q$ & ${S_t}e^{(r - q)(T-t)}$ \\
        Foreign exchange & ${X_t}e^{(r_\$ - r_f)(T-t)}$ \\
        \bottomrule
    \end{tabular}
\end{table}

The forward price is decided under various conditions, those being listed in
Table~\ref{tab:forwardprices}. Given the forward price, we derive the value of
being long a forward contract as 
\begin{equation}
    V_K(t, T) = (F(t, T) - K)e^{-r(T-t)}.
\end{equation}

As a final note, this chapter begins to discuss proof methods for deriving the
values of derivatives. It's fairly instructional to review these as they form
the basis for evaluating many assets, so we shall include the definitions and
some proof examples below.

\begin{definition}
    A \emph{replication proof} is one where we prove the value of an asset by
    demonstrating that if two portfolios always have the same value at time $T$ and
    if neither add/subtract anything of non-zero value between $t$ and $T$, then
    they must have the same value at $t \leq T$.
\end{definition}

\begin{definition}
    \emph{Arbitrage} is a situation where, starting with an empty portfolio, a
    sequence of simple market transactions will end up with a positive portfolio
    value at time $T$.
\end{definition}

\begin{definition}
    A \emph{no-arbitrage proof} uses the assumption that no arbitrage situations
    exist, i.e.\ all market transactions have an equivalent exchange.
\end{definition}

\begin{remark}
    Most no-arbitrage proofs are constructed similarly to a proof by
    contradiction. 
\end{remark}

It's often informative constructing no-arbitrage proofs since they force you to
think about situations in which you can exploit an arbitrage if detected. We
demonstrate no-arbitrage proofs and replication proofs in the following
examples.

\begin{example}
    Let $S_t$ be the current stock of a price paying no income. Let $r_{BID}$ be
    the interest rate at which one can invest/lend money, and $r_{OFF}$ be the
    interest rate at which one can borrow money, $r_{BID} \leq r_{OFF}$. Both
    rates are continuously compounded. Using arbitrage arguments, find upper and
    lower bounds for the forward price of the stock for a forward contract with
    maturity $T > t$.
\end{example}

\begin{solution}
    Assume that $F(t, T) > {S_t}e^{r_{OFF}(T-t)}$. At time $t$, we could short a
    forward contract with forward price $F(t, T)$ (which is done at no upfront
    cost since $V_{F(t, T)}(t, T) = 0$), take out a loan for $S_t$ at rate
    $r_{OFF}$ for time $T - t$, and purchase the asset for $S_t$. Then at time
    $T$, we would execute the forward contract, selling the stock for $F(t, T)$
    and paying back the loan with appreciated value ${S_t}e^{r_{OFF}(T-t)}$,
    leaving us with $F(t, T) - {S_t}e^{r_{OFF}(T-t)} > 0$. This provides an
    upper bound under the assumption of no-arbitrage.

    Assume that $F(t, T) < S_{t}e^{r_{BID}(T - t)}$. At time $t$, we could long
    a forward contract with forward price $F(t, T)$, borrow the asset $S_t$,
    sell the asset and invest/lend the remains at the rate $r_{BID}$ for time $T
    - t$. Then at time $T$, we receive $S_{t}e^{r_{BID}(T - t)}$ at the maturity
    of the deposit, which we can can use $F(t, T)$ of to execute the forward
    contract and return the underlying asset. This leaves us with $
    S_{t}e^{r_{BID}(T - t)} - F(t, T) > 0$. This provides a lower bound under
    the assumption of no-arbitrage.

    Our bounds then come out to 
    \begin{equation}
        S_{t}e^{r_{BID}(T - t)} \leq F(t, T) \leq S_{t}e^{r_{OFF}(T - t)}.
    \end{equation}
\end{solution}
